\documentclass{article}
\usepackage{gvv}
\usepackage{gvv-book}
\begin{document}

\section{Seventh International Olympiad, 1965}

\subsection*{1965/1.}
Determine all values $x$ in the interval $0 \leq x \leq 2\pi$ which satisfy the inequality ${2 \cos x \leq \vert \sqrt {1 + \sin{2x}}}$ - $ \sqrt{1 - \sin{2x}} \vert \leq \sqrt{2}$.

\subsection*{1965/2.}
Consider the system of equations
\[
\begin{align}
a_{11}x_1 + a_{12}x_2 + a_{13}x_3 & = 0 \\
a_{21}x_1 + a_{22}x_2 + a_{23}x_3 & = 0 \\
a_{31}x_1 + a_{32}x_2 + a_{33}x_3 & = 0
\end{align}
\]
with unknowns $x_1, x_2, x_3$. The coefficients satisfy the conditions:
\begin{enumerate}
    \item $a_{11}, a_{22}, a_{33}$ are positive numbers;
    \item the remaining coefficients are negative numbers;
    \item in each equation, the sum of the coefficients is positive.
\end{enumerate}
Prove that the given system has only the solution $x_1 = x_2 = x_3 = 0$.

\subsection*{1965/3.}
Given the tetrahedron $ABCD$ whose edges $AB$ and $CD$ have lengths $a$ and $b$ respectively. The distance between the skew lines $AB$ and $CD$ is $d$, and the angle between them is $\omega$. Tetrahedron $ABCD$ is divided into two solids by plane $e$, parallel to lines $AB$ and $CD$. The ratio of the distances of $e$ from $AB$ and $CD$ is equal to $k$. Compute the ratio of the volumes of the two solids obtained.

\subsection*{1965/4.}
Find all sets of four real numbers $x_1, x_2, x_3, x_4$ such that the sum of any one of them is equal to the sum of the other three and is equal to 2.

\subsection*{1965/5.}
Consider $\triangle OAB$ with acute angle $AOB$. Through a point $M \neq O$, perpendiculars are drawn to $OA$ and $OB$, the feet of which are $P$ and $Q$ respectively. The point of intersection of the altitudes of $\triangle OPQ$ is $H$. What is the locus of $H$ if $M$ is permitted to range over (a) the side $AB$, (b) the interior of $\triangle OAB$?

\subsection*{1965/6.}
In a plane a set of $n$ points ($n \geq 3$) is given. Each pair of points is connected by a segment. Let $d$ be the length of the longest of these segments. We define a diameter of the set to be any connecting segment of length $d$. Prove that the number of diameters of the given set is at most $n$.

\end{document}
